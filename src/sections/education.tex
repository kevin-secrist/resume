\sectionTitle{Education}{\faGraduationCap}
\begin{scholarship}
  \experience
    {Spring 2013}   {Bachelor of Science, Computer Science}{University of Virginia}
    {Fall 2009}
                    {
                        \textbf{School of Engineering and Applied Science}

                        Overall GPA 3.63, CS GPA 3.92 | Dean's List 5 Semesters 

                        \ifcv
                        \textcolor{accentcolor}{Related Coursework}
                        \textit{Mathematics and Formal Logic}: Calculus I-III, Differential Equations, Linear Algebra, Probability, Discrete Mathematics, Theory of Computation, Artificial Intelligence
                        \textit{Software Development}: Software Development Methods, Program \& Data Representation, Web \& Mobile Systems, Computer Architecture, Operating Systems, Malware, Algorithms, Cloud Computing, Databases
                        \fi
                    }
                    {}
\end{scholarship}

\ifcv
\begin{scholarship}
  \experience
    {Spring 2013}  {Consulting Experience}{Service-Learning Practicum}
    {Fall 2012}    {
                        Fourth year two-semester course which gave me an in-depth experience with software development consulting. Worked together as part of a team of six plus a mentor to write software for a local non-profit organization and school, the Virginia Institute of Autism (VIA).

                        \vspace{0.5em}

                        Purpose of the project was to simplify the daily assignment of teachers to students for six periods and seven classrooms during the day, considering a myriad of restrictions and requirements. This manual process cost precious time for the lead teachers, and the project was meant to automate and optimize this.

                        \vspace{0.5em}

                        Project included full SDLC, including requirements gathering, design, agile methodology, scrum, and deep research into domain-specific knowledge related to the scheduling workflow of VIA. The solution included a WAMP stack, the CakePHP MVC framework, normalized relational database schema, and constraint rules implemented leveraging the Java-based AI constraint solver OptaPlanner. Effectively reduced time spent on daily scheduling from an hour to ten minutes. Culminated in a thesis written on the project, design decisions, and potential improvements.
                    }
                    {PHP,Apache,MySQL,CakePHP,OptaPlanner}
\end{scholarship}
\fi
